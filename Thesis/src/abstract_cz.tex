\section*{Abstrakt}
\vspace{0.5cm}

Tato práce se zaměřuje na vývoj auta na dálkové ovládání poháněného STM32 mikrokontroléry. Popisuje stavbu jak auta, tak i vysílačky, představuje použité hardwarové části a jejich využití a~podrobně vysvětluje řídicí software.

Řídicí systém byl od základu vyvinut na starém podvozku Tamiya a tvoří ho na míru vyrobená ovládací deska s moduly a nezbytný řídicí software. Podvozek byl upraven a vybaven výkonnějším BLDC motorem. Aby bylo možné modernizovat vysílačku, byla navržená vlastní deska plošných spojů. Kromě základních funkcí je vůz vybaven různými senzory a moduly s~možností záznamu dat na micro SD kartu. Stav vozu je také nahlašován zpět do vysílačky a~zobrazován na OLED displeji.

Vyvinutý RC systém je v závěru práce otestován v reálných podmínkách. Kromě toho je změřen maximální kontrolovatelný dosah a maximální rychlost.

\vspace{1cm}
\noindent{\bf Klíčová slova:} STM32, RC auto, řídicí software, vestavěný systém, retrofitting

\noindent{\bf Překlad názvu:} Vývoj RC auta řízeného pomocí STM32 mikrokontrolerů
